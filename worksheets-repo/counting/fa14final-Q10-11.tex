\documentclass[11pt]{article}
\usepackage{amsmath,textcomp,amssymb,geometry,graphicx,enumerate}
\usepackage{algorithm} % Boxes/formatting around algorithms
\usepackage[noend]{algpseudocode} % Algorithms
\usepackage{hyperref}
\usepackage{xcolor}
\hypersetup{
    colorlinks=true,
    linkcolor=blue,
    filecolor=magenta,      
    urlcolor=blue,
}


\title{CS70 Fall 2014 Solutions}
\author{Seenu Madhavan}

\pagestyle{myheadings}
\date{November 7, 2020}

\newenvironment{qparts}{\begin{enumerate}[{(}a{)}]}{\end{enumerate}}
\def\endproofmark{$\Box$}
\newenvironment{proof}{\par{\bf Proof}:}{\endproofmark\smallskip}

\textheight=9in
\textwidth=6.5in
\topmargin=-.75in
\oddsidemargin=0.25in
\evensidemargin=0.25in


\begin{document}
\maketitle



\section*{10. Two Face}
Suppose you have two coins, one that has heads on both sides and another that has tails on both sides.
\begin{qparts}
\item
You pick one of the two coins uniformly at random and you flip that coin 400 times. Approximate the
probability of getting more than 220 heads. Your answer should be a number that approximates this
probability, accurate to 2 digits after the decimal point.

\color{blue}
Hint: What is the outcome when we pick the double-headed coin? The double-tailed coin?

\color{black}
\item
You pick one of the two coins uniformly at random and flip it. You repeat this process 400 times, each
time picking one of the two coins uniformly at random and then flipping it, for a total of 400 flips.
Approximate the probability of getting more than 220 heads. Your answer should be a number that
approximates this probability, accurate to 2 digits after the decimal point.

\color{blue}
Hint: How many SDs from the mean is 220?

\color{black}
\item
Now you pick one of the two coins uniformly at random and flip it four times. You repeat this process
100 times, each time picking one of the two coins uniformly at random and then flipping it four times,
for a total of 400 flips. Approximate the probability of getting more than 220 heads. Your answer
should be a number that approximates this probability, accurate to 2 digits after the decimal point.

\color{blue}
Hint: Again, we are estimating a probability, so use CLT.

\color{black}

\end{qparts}









\section*{11. Estimating $\pi$}
One can estimate $\pi$ by playing darts with a special dartboard shown in figure 1. Assume every time you
throw a dart, the dart will always be inside the square. The probability that your dart lands inside the circle
is equal to the ratio of the area of the circle to the area of the square, i.e., $\frac{\pi}{4}$
. Let $X_i$ be the random variable
denoting whether your dart is within the circle after your $i$-th throw.


\noindent
\textbf{How can you estimate $\pi$ using this experiment? How many times should you throw to ensure your estimation error is within 0.01 with probability at least 95\%?} (You can just leave the numerical expression
of the number of times but not compute the exact value.)

\color{blue}

\noindent
Hint: Which inequality allows us to bound the probability that a random variable is a certain distance from the mean?


\end{document}
